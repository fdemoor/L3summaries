\documentclass[a4paper, 10pt]{article}
\usepackage{fullpage}
\usepackage[utf8]{inputenc}
\usepackage{graphicx}
\usepackage{listings}
\usepackage[russian, english]{babel}

\usepackage{amsmath}
\usepackage{amssymb}


\newsavebox\codeone
\newsavebox\codetwo


\title{Images~: OpenGL}
\date{26 avril 1986}
\author{Corentin Ferry\footnote{\texttt{cferr@openmailbox.org}}}

\addto\captionsenglish {
    \renewcommand{\contentsname}{Sommaire}
}
\newcommand{\code}[1]{\texttt{#1}}

\begin{document}
\maketitle

\begin{center}
\includegraphics[width=4cm]{images/OpenGL_logo}
\end{center}

\tableofcontents

\section{Introduction à OpenGL}

Cette section, un peu inutile pour l'évaluation, est à écrire en dernier. 
Éventuellement, on peut insérer une petite explication sur le principe des
fonctions C qu'OpenGL implémente (\code{glSphered} par exemple...).

\section{Modélisation}
\subsection{Modélisations surfacique, volumique}

\paragraph{Modélisation volumique}
La modélisation volumique, passons très rapidement dessus. Elle n'est pas 
utilisée dans le cadre du rendu 3D, mais plutôt en CAD. 

Un volume se décrit par exemple géométriquement par induction, avec~:
\begin{itemize}
    \item L'ensemble de base constitué de volumes simples, du style sphères,
    parallélépipèdes rectangles, etc, que l'on sait construire dans le monde
    du discret;
    \item Des fonctions d'assemblage comme l'union, l'intersection, la 
    différence symétrique...
\end{itemize}

On appelle ceci de la \textbf{constructive solid geometry}.

Si on travaille en discret, on peut aussi représenter un volume par des
\textbf{voxels}, ce qui signifie "volume pixels", à condition d'avoir subdivisé
l'espace en petits cubes. On choisit alors de quel matériau un cube est rempli.

\paragraph{Modélisation surfacique}
Lorsque l'on travaille sur des cartes graphiques : on n'a pas besoin de plonger
à l'intérieur d'un objet que l'on voit de l'extérieur (la plupart du temps),
donc une modélisation surfacique suffit. Cela évite un nombre assez conséquent
de calculs inutiles.

On représente les surfaces de manière discrète, par des polygones, lesquels
peuvent être représentés par des \textbf{triangles} (cf. Delaunay), lesquels 
ne sont jamais que des listes de points. En d'autres termes, il suffit d'avoir 
une liste de points suffisamment bien ordonnée pour représenter une surface.

L'ordre de la liste de points induit aussi \textbf{l'orientation de la surface} 
(celle dont on déduit le vecteur normal). Cette définition est implicite.

Une carte graphique ne sait en fait manipuler que des triangles.
Il va de soi que la qualité du rendu dépend du nombre de triangles en jeu.

\paragraph{Partage des points dans la construction des facettes}
Intuitivement, on va s'éviter de dupliquer les points utilisés pour la 
construction d'une surface. Ainsi, on préfèrera le code de droite au code
de gauche.

\begin{figure}
\begin{lrbox}{\codeone}
\begin{minipage}{0.5\columnwidth}
\begin{lstlisting}
fc1 = (, $P_1$, $P_2$);
fc2 = ($P_1$, $P_2$, $P_3$);
fc3 = ($P_2$, $P_3$, $P_4$);

surf = (fc1, fc2, fc3);

\end{lstlisting}
\end{minipage}
\end{lrbox}

\begin{lrbox}{\codetwo}
\begin{minipage}{0.5\columnwidth}
\begin{lstlisting}
fc1 = (, $P_1$, $P_2$);
fc2 = ($P_1$, $P_2$, $P_3$);
fc3 = ($P_2$, $P_3$, $P_4$);

surf = (fc1, fc2, fc3);
\end{lstlisting}
\end{minipage}
\end{lrbox}

{\usebox\codeone}
{\usebox\codetwo}


\caption{Deux codes, qui montrent ce qu'il faut faire et ce qu'il ne faut
    pas faire}
\end{figure}

\subsection{Interpolation de surfaces}

\paragraph{Courbes de Bézier}
Déjà, si la question est posée : Bézier n'était pas un mathématicien ! C'était
un ingénieur chez Citroën. Pour que Citroën puisse produire de belles caisses 
(comment voulez-vous vendre ?), il inventa des courbes à base de polynômes
de Bernstein (\foreignlanguage{russian}{Сергей Натанович Бернштейн}), 
russe\footnote{Je précise qu'il a étudié à Supélec et à la Sorbonne !}.

Un petit rappel de la formule du $k$-ième polynôme de degré $n$ de Bernstein : 

$$
B_k^n(x) = \binom{n}{k} x^k \left(1-x\right)^{n-k} \text{ pour } x \in [0,1]
$$

Une surface est alors définie par la double somme de Bézier~:

$$
S(x, y) = \sum_{i=1}^{n_x}\sum_{j=1}^{n_y} P_{i,j} B_i^{n_x}(x) B_j^{n_y}(x)
\text{ pour } x,y \in [0,1]
$$
où les $P_i,j$ sont les points de passage dits \textbf{points de contrôle}.
Il faut faire l'homothétie pour transformer ça en une surface définie ailleurs
que sur $[0,1]^2$.

Lors du rendu, la surface de Bézier est discrétisée à un certain pas, dont
dépend le nombre de triangles fabriqué, et donc le temps de rendu.

\section{Rendu}

\section{Animation}

\end{document}
