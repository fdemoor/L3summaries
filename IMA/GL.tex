\documentclass[a4paper, 10pt]{article}
\usepackage{fullpage}
\usepackage[utf8]{inputenc}
\usepackage{graphicx}

\title{Images~: OpenGL}
\date{26 avril 1986}
\author{jean-marie}

\renewcommand{\contentsname}{Sommaire}
\newcommand{\code}[1]{\texttt{#1}}

\begin{document}
\maketitle

\begin{center}
\includegraphics[width=4cm]{images/OpenGL_logo}
\end{center}

\tableofcontents

\section{Introduction à OpenGL}

Cette section, un peu inutile pour l'évaluation, est à écrire en dernier. 
Éventuellement, on peut insérer une petite explication sur le principe des
fonctions C qu'OpenGL implémente (\code{glSphered} par exemple...).

\section{Modélisation}

\section{Rendu}

\section{Animation}

\end{document}
