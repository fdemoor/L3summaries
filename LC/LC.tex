%summary for LC

\documentclass[french]{article}
\usepackage[utf8]{inputenc}
\usepackage[T1]{fontenc}
\usepackage{babel}
\usepackage{lmodern}
\usepackage{graphicx}
\usepackage{tikz}

\usepackage{amsmath}
\usepackage{amsfonts}

\title{Logique}
\date{}
\author{L3 RI}


\begin{document}
\maketitle
\tableofcontents
\section{Induction}
Définition inductive : ensemble de base, des fonctions constructrices d'autres éléments. L'ensemble est alors défini comme le plus petit contenant la base et stable par les constructeurs.

Tout élément s'obtient alors depuis des éléments de base et un nombre fini de constructeurs. Représente sous forme d'arbre $\rightarrow$ hauteur d'un élément.

Pour prouver quelque chose, le faire sur la base et montrer la stabilité par les constructeurs.

Définition non-ambigüe : il existe un unique arbre d'induction pour chaque élément. Condition nécessaire et suffisante pour l'existence de fonctions définies inductivement.

\section{Calcul propositionnel}
\subsection{Syntaxe et sémantique}
Propositions définies par induction : variables et connecteurs. Non ambigu.

Valuation : fonction qui assigne à chaque variable 0 ou 1.
Les valuations sont prolongeables sur l'ensemble des formules (en uitilisant la sémantique des connecteurs). Formule $F$ satisfaite par la valuation $\varphi$ : $\varphi (F) = 1$.

$F$ et $G$ formules sont équivalentes \emph{ssi} $\forall\varphi ,\varphi F) = \varphi (G)$. On note $F\equiv G$. 

Soit $\Sigma$ un ensemble de formules, $F$ une formule. $F$ conséquence de $\Sigma$ ssi toute valuation qui satisfait $\Sigma$ satisfait $F$, noté $\Sigma\models F$ (aussi équivalent à $\Sigma\cup F$ non satisfiable. $\Sigma$ satisfait par une valuation si celle-ci satisfait toutes ses formules.

Substitution d'une variable $p$ par une formule $G$ dans une formule $G$ noté $F[G/p]$. Opération syntaxique.

Support d'une formule : ensemble des variables qu'elle utilise.


\subsection{Formes normales}
Pour chaque table de vérité, on peut trouver une formule dont c'est la table.

Toute formule est équivalente à une formule sous forme normale disjonctive (FND) : union d'intersections de littéraux (littéral = variable ou négation d'une variable). De même, toute formule équivalente à une forme normale conjonctive (FNC).

Système complet de connecteur : ensemble de connecteurs qui permettent d'engendrer toutes les formules (sémantiquement). Ex : $\{nand\}$.

\paragraph{Théorème de compacité}
Ensemble $\Sigma$ de formules. $\Sigma$ satisfiable ssi $\Sigma$ finiment satisfiable. Et $\Sigma$ est finiment satisfiable si tout sous-ensemble fini de $\Sigma$ est satisfiable.

Corollaire : $\Sigma\models F$ ssi il existe un sous-ensemble fini de $\Sigma$ dont $F$ est conséquence.

\subsection{Systèmes de déduction}
Système formel : alphabet, procédé de formation des formules, ensemble d'axiomes et ensemble de règles de déduction. Si on déduit $F$ de $\Sigma$ avec ces règles (syntaxiques), on note $\Sigma\vdash F$.

Un système de déduction est correct si $\Sigma\vdash F$ (syntaxique) implique $\Sigma\models F$ (sémantique).
Un système de déduction est complet si  $\Sigma\models F$ (sémantique) implique  $\Sigma\vdash F$ (syntaxique).

\subsection{Système de déduction naturelle}
\subsubsection{Logique minimale (NM)}
$\frac{}{\Gamma,A\vdash A}$Ax\\

$\frac{\Gamma ,A\vdash B}{\Gamma\vdash A\implies B}$ intro $\implies$ 
$\frac{\Gamma\vdash A, \Delta\vdash A\implies B}{\Gamma ,\Delta\vdash B}$ elim $\implies$\\

%TO DO !!
À compléter, j'ai la flemme là...
%TO DO !!

\subsubsection{Logique intuitionniste (NJ)}
On ajoute le symbole $\bot$, on ajoute une règle :
$$\frac{\Gamma\vdash\bot}{\Gamma\vdash A}\text{elim $\bot$}$$
On enlève le connecteur $\neg$ et on en fait une macro : $\neg A$ = $A\implies\bot$.

%TO DO !!
à compléter : elim intro non
%TO DO !!

\subsubsection{Logique classique (NK)}
On ajoute le Tiers exclus :
$$\frac{}{\Gamma\vdash A\vee\neg A}\text{Tiers exclus}$$

NK plus forte que NJ plus forte que NM. On est plus fort quand on permet démontrer au moins les mêmes choses.
 
\subsection{Traduction de logiques}
Soit $\mathcal{L}$ plus forte que $\mathcal{L}'$. Une traduction est une fonction $\varphi$ qui transforme les formules : si $\vdash F$ dans $\mathcal{L}$ alors $\vdash\varphi (F)$ dans $\mathcal{L}'$. 

%TO DO !!
HELP j'ai l'impression que c'est l'inverse. Je suis peut etre juste trop fatigué
%TO DO !!

\subsection{Déduction par coupure}
On s'intéresse aux clauses : disjonction de littéraux. Toute formule peut être mise sous forme de conjonction de clauses (FNC).

On met les clauses sous la forme $(a_1\wedge\dots\wedge a_n)\implies (b_1\vee\dots\vee b_m)$ où les $a_i$ sont les variables négatives et les $b_i$ les positives. On note aussi une clause $C=(\Gamma,\Delta)$ où $a_i\in\Gamma$ et $b_i\in\Delta$.

Règle de coupure :
$$\frac{(\Gamma_1,\Delta_1\cup\{p\})\ (\Gamma_2\cup\{p\}, \Delta_2)}{(\Gamma_1\cup\Gamma_2, \Delta_1\cup\Delta2)}$$

Déduction par coupure correcte et complète.

Idée : On fait toutes les coupures sur une variable (on obtient le \emph{résolvant}). Si un résolvant est nul, l'ensemble de clauses était non satisfiable.

\paragraph{Théorème :} Tout ensemble de clause non satisfiable possède une réfutation par coupure.

\subsection{Calcul des séquents}
Un séquent : 2 suites finies de formules, noté  $(H_1,\dots , H_n)\vdash (C_1,\dots , C_m)$ qui signifie $(H_1\wedge\dots\wedge H_n)\implies (C_1\vee\dots\vee C_m)$

\subsubsection{Calcul des séquents classique : LK}
%TO DO!!
AJOUTER LES REGLES

\paragraph{Théorème fondamental (Hauptsatz)}
Pour toue preuve dans LK, il existe une preuve de même conclusion sans utiliser la règle de coupure.

\paragraph{Cohérence} Un système de déduction est cohérent s'il ne permet pas de prouver $A$ et $\neg A$ sans hypothèses. Cohérent ssi on ne peut pas prouver le séquent vide.

LK cohérent (grâce au Hauptsatz).

\subsubsection{Calcul des séquents intuitionniste (LJ)}
Restriction des séquents : une seule formule au plus à droite de $\vdash$.

Introductiçon du ou:
$\frac{\Gamma\vdash A}{\Gamma\vdash A\vee B}\ \frac{\Gamma\vdash B}{\Gamma\vdash A\vee B}$.

\section{Calcul des prédicats}

\end{document}
